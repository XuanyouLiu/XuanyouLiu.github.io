% !TEX program = xelatex
% !TEX encoding = UTF-8 Unicode
% 中文简历 - 刘炫佑
% 使用 XeLaTeX 编译

\documentclass{resume}

\begin{document}
\pagenumbering{gobble}

\name{刘炫佑 (Xuanyou Liu)}

\basicInfo{
  Evanston, IL \quad | \quad
  (+1) 267-809-0364 \quad | \quad
  \email{xuanyou@u.northwestern.edu} \quad | \quad
  \homepage[xuanyouliu.com]{https://xuanyouliu.com}
}

%-----------------------------------------------------------------------
\section{教育背景}

\datedsubsection{\textbf{西北大学 (Northwestern University)},美国伊利诺伊州}{2025.09 -- 2030}
计算机科学博士 \quad 导师:Karan Ahuja 教授
\begin{itemize}
  \item 研究方向:人机交互 (HCI)、触觉技术
  \item 荣誉:TGS First-Year Fellowship
\end{itemize}

\datedsubsection{\textbf{宾夕法尼亚大学 (University of Pennsylvania)},美国宾夕法尼亚州}{2023.09 -- 2025.05}
机器人学硕士 \quad GPA: 4.00/4.00,排名: 1/64
\begin{itemize}
  \item 荣誉:GAPSA Professional Student Travel Award (2025)
\end{itemize}

\datedsubsection{\textbf{西安交通大学},中国陕西省}{2019.09 -- 2023.07}
工业设计学士 \quad GPA: 3.86/4.30,排名: 1/21
\begin{itemize}
  \item 荣誉:优秀毕业生、优秀学生干部
\end{itemize}

%-----------------------------------------------------------------------
\section{学术论文}

\datedsubsection{\textbf{Seeing with the Hands: A Sensory Substitution That Supports Manual Interactions}}{CHI 2025}
Shan-Yuan Teng*, Gene Kim*, \textbf{Xuanyou Liu*}, Pedro Lopes \quad (*共同一作)\\
{\small ACM Conference on Human Factors in Computing Systems (CHI) \href{https://doi.org/10.1145/3706598.3713785}{\color{NUPurple}[链接]}}

\datedsubsection{\textbf{TacTex: A Textile Interface with Seamlessly-Integrated Electrodes for Capacitive Pressure Sensing}}{CHI 2024}
Hongnan Lin, \textbf{Xuanyou Liu}, Shengsheng Jiang, Qi Wang, Ye Tao, et al.\\
{\small ACM Conference on Human Factors in Computing Systems (CHI) \href{https://doi.org/10.1145/3613904.3642063}{\color{NUPurple}[链接]}}

%-----------------------------------------------------------------------
\section{研究项目}

\datedsubsection{\textbf{基于对比学习的鲁棒电阻抗断层成像 (EIT) 手部追踪}}{2025.09 -- 至今}
\begin{itemize}
  \item 设计基于 Transformer 的对比学习模型,实现跨用户、跨手臂姿态的连续手势追踪
  \item 设计紧凑型 PCB 并使用 RTOS 实现高频生物阻抗传感的实时固件
\end{itemize}

\datedsubsection{\textbf{可穿戴触觉设备的紧凑型电触觉模块}}{2024.08 -- 2024.12}
\begin{itemize}
  \item 开发小型化电触觉刺激模块,用于可穿戴设备的高保真触觉反馈
  \item 设计定制 PCB 和固件,实现实时控制并与 VR 环境集成
\end{itemize}

\datedsubsection{\textbf{Ins-Bucks 3D 食品打印机}}{2021.06 -- 2021.07}
\begin{itemize}
  \item 设计基于昆虫蛋白的 3D 食品打印服务生态系统,整合硬件、UI 和服务蓝图
  \item 荣获\textbf{最具创新奖},在可持续蛋白质个性化营养领域获得认可
\end{itemize}

%-----------------------------------------------------------------------
\section{工作经历}

\datedsubsection{\textbf{陆逊梯卡华宏眼镜有限公司},中国东莞}{2020.07 -- 2020.09}
研发实习生
\begin{itemize}
  \item 优化失蜡铸造工艺的温度参数,将缺陷率降低 15\%
  \item 重构生产线 PLC 的梯形逻辑代码,提高生产周期效率
\end{itemize}

%-----------------------------------------------------------------------
\section{专业技能}

\begin{itemize}[parsep=0.2ex]
  \item \textbf{研究领域:}人机交互 (HCI)、触觉技术、机器人学
  \item \textbf{硬件:}PCB 设计 (Altium/KiCad)、Arduino/ESP32、3D 建模 (SolidWorks)
  \item \textbf{编程:}Python, C/C++, MATLAB, LaTeX, PyTorch, RTOS, scikit-learn
\end{itemize}

%-----------------------------------------------------------------------
\section{社会服务}

\begin{itemize}[parsep=0.2ex]
  \item \textbf{助教:}ESE5190 智能设备课程(宾夕法尼亚大学)
  \item \textbf{编程教育志愿者:}Fife-Penn CS Academy,教授 Python/Arduino
  \item \textbf{国际志愿者:}联合国儿童基金会 (UNICEF)、国际劳工组织 (ILO) 项目
\end{itemize}

\end{document}
